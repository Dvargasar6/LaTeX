\documentclass{article}
\usepackage{graphicx}
\usepackage{listings}
\usepackage{xcolor}


\definecolor{codegreen}{rgb}{0,0.6,0}
\definecolor{codegray}{rgb}{0.5,0.5,0.5}
\definecolor{codepurple}{rgb}{0.58,0.0.82}
\definecolor{backcolor}{rgb}{0.95,0.95,0.95}

\lstdefinestyle{python_style}{
backgroundcolor = \color{backcolor},
commentstyle = \color{codegreen},
keywordstyle = \color{magenta},
numberstyle = \tiny \color{codegray},
stringstyle = \color{codepurple},
basicstyle = \ttfamily \footnotesize,
breaklines = true,
captionpos = b
}

\lstdefinestyle{python_style}{
  language=Python,
  backgroundcolor=\color{backcolor},
  basicstyle=\ttfamily\small,
  keywordstyle=\color{magenta}\bfseries,
  stringstyle=\color{codepurple},
  commentstyle=\color{codegreen}\itshape,
  identifierstyle=\color{black},
  showstringspaces=false,
  frame=single,
  breaklines=true,
  tabsize=4,
  numbers=none,
  captionpos=b
}


\lstdefinestyle{java_style}{
    language=Java,
    basicstyle=\ttfamily\small,
    keywordstyle=\color{blue}\bfseries,
    stringstyle=\color{orange},
    commentstyle=\color{gray}\itshape,
    backgroundcolor=\color{white},
    frame=single,
    rulecolor=\color{gray},
    tabsize=4,
    showspaces=false,
    showstringspaces=false,
    breaklines=true,
    captionpos=b
}

\title{CODES IN \LaTeX}
\author{}
\date{}

\begin{document}

\maketitle

It can add a format code with "verb": \verb|code style|

\vspace{10pt}

With "verbatim" it can begin a code style without format. Verbatim allows you to write in the .tex file as it is in the code, respecting spaces, lines and tabs:

\begin{verbatim}
import numpy as np
import matplotlib.pyplot as plt

for i in range 10:
    print(i)
\end{verbatim}

\vspace{10pt}

With "lstlisting" it can create a format for add to code to text:

Code in Python:

\begin{lstlisting}[language = Python, caption = {Code in Python}, style = python_style]
# Code in LaTeX           
import numpy as np
import matplotlib.pyplot as plt

celsius = np.array([-40,-10,0,8,15,22,38], dtype = float)
fahrenheit = np.array([-40,14,32,46,59,72,100], dtype = float)

for i in range(0,len(celsius)):
    print(f'{celsius[i]} °C = {fahrenheit[i]} °F')
    
\end{lstlisting}

Code in Java:

\begin{lstlisting}[style=java_style, caption={Code in Java}]
package com.Notas;
public class Principal {
    public static void main(String[] args) {
        VentanaPrincipal ventana_principal = new VentanaPrincipal();
        ventana_principal.setVisible(true);
    }
}
\end{lstlisting}

\end{document}
